\documentclass{article}

\usepackage{graphics}
\usepackage{graphicx}
\usepackage{color}
\usepackage{amssymb}
\usepackage{amsmath}

\newcommand\vect[1]{\mathbf{#1}}
\newcommand\x{\vect{x}}
\newcommand\dx{\vect{\dot{x}}}
\newcommand\ddx{\vect{\ddot{x}}}
\newcommand\reals{\mathbb{R}}
\newcommand\trans{\mathbf{t}}
\newcommand\dtrans{\mathbf{\dot{t}}}

\title{Rigid-body motion}
\author{Florent Lamiraux}
\date{}

\begin{document}
\maketitle

\section{Rotation}

A rotation is a rigid-body tranformation that keeps the origin unchanged. A rotation can be
represented by a $3\times3$ matrix $R$ satisfying:
\begin{equation}\label{eq:rotation-matrix}
RR^T = I_3 \ \mbox{and}\ \det(R) = 1
\end{equation}
where $I_3$ is the identity matrix.
The set of rotation is called \textit{Special orthogonal group} and is denoted by $SO(3)$.

\subsection{rotation motion}

Let us consider a mapping $R$ from $\reals$ to $SO(3)$ representing a time-varying rotation.
A point $\vect{u}\in\reals^3$ is mapped at time $t$ to $R(t)\vect{u}$. The velocity at time $t$
of this point is given by $\dot{R}(t)\vect{u}$.

Let us differentiate~(\ref{eq:rotation-matrix}):
\begin{eqnarray}
  \dot{R} R^T + R \dot{R^T} = 0 \\
  \dot{R} R^T + (\dot{R}R^T)^T = 0
  \label{eq:dotR}
\end{eqnarray}
The latter equality states that $\dot{R}R^T$ is a skew-symmetric matrix.

Let $\omega=(\omega_1,\omega_2,\omega_3)\in\reals^3$ be a vector. We denote by
\begin{equation}\label{eq:skew-symmetric}
  \left[\omega\right]_{\times} = \left(\begin{array}{ccc}
    0 & -\omega_3 & \omega_2 \\
    \omega_3 & 0 & -\omega_1 \\
    -\omega_2 & \omega_1 & 0
  \end{array}\right)
\end{equation}
Let us notice that for any $\omega$ and any $\vect{u}$ in $\reals^3$,
$$
\left[\omega\right]_{\times}\vec{u} = \omega\times\vec{u}.
$$
where $\times$ denotes the cross product.
From~(\ref{eq:dotR}), we can state that there exists $\omega\in\reals^3$ such that
\begin{eqnarray*}
\dot{R}R^T &=& \left[\omega\right]_{\times}\\
\dot{R} &=& \left[\omega\right]_{\times} R\\
\end{eqnarray*}
The velocity of point $\vect{u}$ defined above along the rotation motion is therefore given by
$$
\dot{R}(t)\vect{u} = \left[\omega\right]_{\times} R\,\vect{u} = \omega\times(R\,\vect{u}).
$$
which is the well known formula giving the velocity of a point moving on an object with angular velocity $\omega$.

\subsection{Exponential}

Let us consider a rotation motion $R(t)$ starting at $I_3$ and with constant angular velocity $\omega\in\reals^3$:
\begin{eqnarray*}
  R(0) &=& I_3\\
  \dot{R}(t) &=& \left[\omega\right]_{\times} R(t)
\end{eqnarray*}
It is easy to state that the solution to this differential equation is given by
$$
R(t) = \exp(t \left[\omega\right]_{\times}) = \sum_{i=0}^{\infty}\frac{1}{i!}t^i \left[\omega\right]_{\times}^i
$$

\subsection{Logarithm}

The $\exp$ mapping from the space of skew-symmetric matrices to $SO(3)$ is surjective. This means that for any $R\in SO(3)$, there exists $\omega\in\reals^3$ such that
$$
R = \exp(\left[\omega\right]_{\times})
$$
we define $\log(R)$ as the skew-symmetric matrix of smaller norm among all solutions of the above equation.

By extension, we will sometimes confuse the skew-symmetric matrix with the underlying vector and write
$$
\omega = \log(R)\ \mbox{instead of}\ \left[\omega\right]_{\times} = \log(R)
$$

\section{Rigid-body tranformation}

A rigid-body tranformation $M$ is a mapping from $\reals^3$ to itself that preserves distances and angles. Any rigid-body tranformation can be expressed as:
$$
\forall \vect{p}\in\reals^3,\ M(\vect{p}) = R\,\vect{p} + \trans
$$
where $R\in SO(3)$ is a rotation matrix and $\trans$ is a vector.

The space of rigid-body tranformations is called the \textit{special euclidean group} and is denoted by $SE(3)$.

\subsection{Homogeneous matrix}

A rigid-body tranformation can be represented by a $4\times4$ matrix as follows:
$$
H = \left(\begin{array}{cc}
  R & \trans\\
  0 & 1
\end{array}\right).
$$
By adding a one as the fourth componentof $\vect{p}$, the rigid-body tranformation can be represented by a matrix-vector product as follows:
$$
\left(\begin{array}{c}M(\vect{p})\\1\end{array}\right) = H\left(\begin{array}{c}\vect{p}\\1\end{array}\right) = \left(\begin{array}{c}R\,\vect{p}+\trans\\1\end{array}\right)
$$

$H$ is called a homogeneous matrix.
      
\section{Rigid-body motion}

Let us consider a rigid-body motion represented by a time varying homogeneous matrix $H(t)$.
At time $t$, point $\vect{p}$ is moved to $R(t)\,\vect{p}+\trans(t)$.

The velocity of this point at time $t$ is thus given by
$$
\left(\begin{array}{c}\vect{\dot{p}}\\0\end{array}\right) = \dot{H}(t)\left(\begin{array}{c}\vect{p}\\1\end{array}\right)\\
\ \ \ \mbox{or}\ \ \ \vect{\dot{p}} = \left[\omega\right]_{\times} R\,\vect{p}+\dtrans
$$

$\dtrans$ is the velocity of the image of the origin at time $t$. Let us denote by $\vect{v}$ this velocity expressed in the moving frame:
$$
\vect{v} = R^T\dtrans.
$$
Similarly, let us denote by $\Omega$ the angular velocity expressed in the moving frame:
$$
\Omega = R^T \omega
$$
If we admit that
$$
\left[R^T\omega\right]_{\times} = R^T\left[\omega\right]_{\times} R.
$$
Then, we can write
\begin{equation}\label{eq:diff-SE3}
\dot{H} = \left(\begin{array}{cc}R \left[\Omega\right]_{\times} & R\vect{v}\\0 & 0\end{array}\right)=
  H \left(\begin{array}{cc} \left[\Omega\right]_{\times} & \vect{v}\\0 & 0\end{array}\right)
\end{equation}

If we consider a motion with constant linear velocity of the origin and angular velocity expressed in the moving frame, and starting at identity, Equation~(\ref{eq:diff-SE3}) can be seen as a differential equation in $H$. The solution to this equation is
$$
H(t) = \exp t \left(\begin{array}{cc} \left[\Omega\right]_{\times} & \vect{v}\\0 & 0\end{array}\right)
  = \sum_{i=0}^{\infty}\frac{1}{i!}t^i \left(\begin{array}{cc} \left[\Omega\right]_{\times} & \vect{v}\\0 & 0\end{array}\right)^i
$$

The motion defined by $H(t)$ is called a screw motion. It has the following properties:
\begin{itemize}
\item if $\Omega = 0$, this is a translation of constant velocity $\vect{v}$,
\item otherwise there exists a straight line called the \textit{axis} of the screw motion such that the motion of the points of the axis is a pure translation of constant velocity along the axis.
\end{itemize}

Similarly, we define the logarithm of a rigid-body transformation as the screw motion of minimal norm whose exponential is the rigid-body transformation. For some singular rigid-body transformations, the logarithm may not be uniquely defined.

\section{What you need to remember}

\begin{enumerate}
\item The derivative of a rigid-body motion is a screw represented by the linear and angular velocities of the image of the origin by the rigid-body motion and expressed in the moving frame.
\item the exponential maps screw velocities to rigid-body transformations.
\item The logarithm maps the other way back and is uniquely defined in a neighborhood of $0\in\reals^6$.
\item this mapping defines a distance in $SE(3)$ as follows:
  $$d(M_1,M_2) = \|\log M_1^{-1}M_2 \|.$$
The distance is indeed equal to zero if and only if $M_1=M_2$.
\end{enumerate}

\section{Notations}

Following the properties outlined in the previous sections, we define the
following notation. Let $\mathbf{p}_1, \mathbf{p}_2$ be elements of $SO(3)$
(respectively $SE(3)$):
$$
\mathbf{p}_{2} \ominus \mathbf{p}_{1} = \log(p_{1}^{-1}p_{2})
$$
is the constant velocity in $\reals^3$ (respectively $\reals^6$) that drives
$\mathbf{p}_{1}$ to $\mathbf{p}_{2}$.

Similarly, let $\mathbf{u}\in\reals^3$ (respectively $\reals^6$) be a velocity,
$$
\mathbf{p}_1 \oplus \mathbf{u} = \mathbf{p}_1 \exp\ \mathbf{u}
$$
is the the element of $SO(3)$ (respectively $SE(3)$) reached after following
the constant velocity $\mathbf{u}$ from $\mathbf{p}_1$.

\end{document}
